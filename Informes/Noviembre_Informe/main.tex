
\documentclass{article}
\usepackage[table]{xcolor}  % Required for rowcolor
\usepackage{siunitx}  % Added siunitx package for number formatting
\usepackage{graphicx}
\usepackage{caption}
\usepackage{tocloft}
\usepackage{float}
\usepackage{hyperref}
\usepackage{setspace}
\usepackage{blindtext}
\usepackage{ragged2e}
\usepackage[margin=1in]{geometry}

% Customizing Table of Contents
\renewcommand{\cftsecleader}{\cftdotfill{\cftdotsep}}
\renewcommand{\baselinestretch}{1.5} 

% Custom command for including figures with captions
\newcommand{\vputfigure}[3]{%
  \begin{figure}[H]
    \centering
    \includegraphics[height=4cm]{#1}
    \captionsetup{font=scriptsize}
    \caption{#2}
    \label{#3}
  \end{figure}%
}

\newcommand{\hputfigure}[3]{%
  \begin{figure}[H]
    \centering
    \includegraphics[width=4cm]{#1}
    \captionsetup{font=scriptsize}
    \caption{#2}
    \label{#3}
  \end{figure}%
}



\justifying

\begin{document}
\title{Reporte subida a Aesa Noviembre}
\author{Diego Quispe Cangalaya}
\date{\today}

\maketitle

\tableofcontents
\newpage 

\listoffigures
\newpage 


\section{Resumen}
El objetivo de esta subida fue demostrar que el equipo Edge AIoT Box puede recabar informacion de minimo 15 dias.
Por lo que se debio intervenir la maquinaria R1600H que contiene el mencionado equipo instalado.
\vspace{5mm}
\newline En esta subida se realizaron las siguientes operaciones:
\begin{enumerate}
    \item Coordinacion el ingreso a la mina MINSUR
    \item Renovacion de las capacitacion Anexo 4 y Anexo 5
    \item Obtencion del permiso para intervenir R1600H
    \item Se registro el equipamento traido para intervencion
    \item Se realizaron los documentos requeridos para intervercion
    \item Se reprogramaron los nodos
    \item Se intervino la maquinaria R1600H y se remmplazo el Edge
    \item Se examino la data recoletada por el anterior R1600H
    \item Se enviaron muestras de los datos a la web de CST Peru
    \item Registramos nuestra salida de equipaje
    \item Se realizo una charla del avance a AESA / MINSUR
    \item Se superviso el envio de datos mediante un operador
\end{enumerate}

\newpage 


\section{Capacitaciones e Ingreso a la U.M. San Rafael}
Se tomo una movilidad particular para ir de la ciudad de Juliaca hacia Antauta el dia viernes 5 de Noviembre. 
Nos instalamos en el hotel y realizamos la guia de entrada para el ingreso de herramientas.
\vspace{5mm}
\newline
Al dia siguiente se realizo la induccion para renovar el anexo 4 y 5 para trabajadores temporales lo cual todo el dia.
\vspace{5mm}
\newline
El dia Domingo 5 de Noviembre, se nos informa que el equipo se encontrara disponible para su intervencion en la tarde.
\newpage 


\section{Intervecion del equipo R1600H}
Una vez completados los documentos formales para intervenir el equipo R1600H, se logra establecer conexion con el Edge.
El equipo se encontraba desconectado y en un 
\vspace{5mm}
\newline
Se extrae la base de datos del equipo y nos retiramos de la maquinaria que todavia seguia en labores de mantenimiento.
Se analizo la base de datos:
\begin{enumerate}
    \renewcommand{\labelenumi}{(\roman{enumi})}
    \item Se encontro aproximadamente 3 millones de datos en la base de datos
    \vputfigure{pictures/Edge_sql_aesa.jpg}{Datos de "salud" en la base de datos}{fig:edge_sql_data}

    \item Graficando algunos parametros, se encontro la fechas en que se registraron estos datos: 14 Setiembre hasta 27 de Octubre
    \vputfigure{pictures/Grafica_RPM_Fecha.jpg}{Grafica RPM con respecto a las fechas}{fig:rpm_graph}
    \vputfigure{pictures/Graficas_30_dias.jpg}{Graficas de todos los parametros}{fig:all_graphs}
    
    \item Se valida la existencia de data de mas de 30 dias, se sube a reparto de guardia para comenzar a enviar la data
    \hputfigure{pictures/Nodo_Portable_Banco_de_carga.jpg}{Nodos portables en espera para enviar informacion}{fig:banco_de_carga}

    \item Se valido la llegada de informacion al servidor de CST Peru
    \vputfigure{pictures/web_cst_peru.jpg}{Equipo AE-SC-03 en la primera semana de data}{fig:cst_web}
\end{enumerate}

\newpage 


\section{Estado del equipo previo}
A continuacion se mostrara el estado del Edge AioT Box instalado previamente en la maquinaria R1600H, asi como el estado de los tres nodos portables
instalados en el banco de carga:
\subsection{Edge AioT Box}
  \begin{enumerate}
    \renewcommand{\labelenumi}{(\roman{enumi})}
    \item El equipo Edge AioT Box previamente instalado tiene el cable de conexion salido
    \hputfigure{pictures/Edge_R1600H_cable_conduit_error.jpg}
                {El cable conduit se encuentra fuera de posicion, dejando vulnerable al equipo}
                {fig:edge_fail_conduit}
    
    \item Se extrajo el equipo de la maquinaria, y se noto que el soporte metalico sufrio un doblez
    \hputfigure{pictures/Edge_R1600H_asiento_alto.jpg}
                {Asiento del conductor bajo presion}
                {fig:asiento_r1600h}
    \hputfigure{pictures/Edge_R1600H_Placa_deformada.jpg}
                {Deformacion en la placa metalica}
                {fig:edge_soporte_steel}

    \item Se realizo una prueba en la cabina, se observo que hay una parte del asiento que golpea al equipo al tener un operario sentado
    \hputfigure{pictures/Edge_R1600H_choque_con_asiento.jpg}
                {El equipo Edge se golpea con frecuencia con un elemento metalico}
                {fig:edge_hit_steel}

    \item El equipo se abrio para investigarlo internamente, el equipo Edg no muestra indicios de daños. Solamente se observo 2 tornillos
    sueltos. 
    \vputfigure{pictures/Edge_R1600H_estado_inicial_interior.jpg}
                {Interior del equipo Edge, no se encuentra daños}
                {fig:inside_edge}
    \vputfigure{pictures/Edge_R1600H_tornillos_salidos.jpg}
                {Tornillos sueltos encontrados}
                {fig:bolt_2_edge}
  \end{enumerate}

\subsection{Nodo Portable}
  \begin{enumerate}
    \renewcommand{\labelenumi}{(\roman{enumi})}
    \item Se abrieron los tres nodos portables, y en dos de ellos se encotraron los botones para programacion 
    dañado y la entrada del cable de bateria inclinado hacia el ESP32
    \hputfigure{pictures/Nodo_Portable_fallos_por_presion.jpg}
                {La placa del Nodo Portable y los daños presentes}
                {fig:nodo_damaged}

    \item Se dejaron dos nodos portables instalados con el codigo actualizado, y se realizo una prueba de envio 
    al servidor de CST Peru.
    \vputfigure{pictures/Nodo_Portable_Pruena_6_Noviembre_300_datos.jpg}
                {Los datos llegaron a la web de CST Peru}
                {fig:nodo_test_cst}
  \end{enumerate}

\newpage 
\section{Resumen de gastos}

\begin{table}[htb]
  \centering
  \begin{tabular}{|c|p{4cm}|S[table-format=3.2]|}
    
    \hline
    \rowcolor{black!20} % Background color for the first row
    Fecha & \multicolumn{1}{c|}{Motivo} & {Gasto (\text{Soles s/})} \\
    
    \hline
    02/11/23 & Desayuno & 21\\    % Jueves
    \hline
    02/11/23 & Autobus a Juliaca & 140\\
    \hline
    02/11/23 & Almuerzo & 15\\    
    
    \hline
    03/11/23 & Almuerzo & 25\\    % Viernes
    \hline
    03/11/23 & Cena & 19\\        
    
    \hline
    04/11/23 & Desayuno & 10\\    % Sabado
    \hline
    04/11/23 & Almuerzo & 18\\
    \hline
    04/11/23 & Cena & 19\\
    
    \hline
    04/11/23 & Desayuno & 10\\    % Sabado
    \hline
    04/11/23 & Almuerzo & 18\\
    \hline
    04/11/23 & Cena & 19\\
    
    \hline
    05/11/23 & Desayuno & 8\\    % Domingo
    \hline
    05/11/23 & Medicamentos & 6\\
    \hline
    05/11/23 & Cena & 16\\
    
    \hline
    06/11/23 & Medicamentos & 9\\    % Lunes
    \hline
    06/11/23 & Cena & 18\\    

    \hline
    07/11/23 & Desayuno & 8\\    % Martes
    \hline
    07/11/23 & Cena & 20\\    

    \hline
    08/11/23 & Cena & 20\\    % Miercoles

    \hline
    09/11/23 & Cena & 16\\    % Jueves

    \hline
    \rowcolor{blue!20}
    & Total & \num{435} \\
    \hline
  \end{tabular}
  \caption{Gastos realizados durante la subida a Mina}
  \label{tab:schedule_expenses}
\end{table}

\end{document}
